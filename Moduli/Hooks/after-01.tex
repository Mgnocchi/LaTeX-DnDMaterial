\documentclass[10pt,twoside, twocolumn, openany]{dndbook}

\usepackage[english]{babel}
\usepackage[utf8]{inputenc}
\usepackage{lipsum}
\usepackage{listings}
\usepackage{flushend}
\usepackage{dblfloatfix}
\usepackage[normalem]{ulem}

\lstset{%
  basicstyle=\ttfamily,
  language=[LaTeX]{TeX},
}


\begin{document}

\section{Grok, L'orco}
Grok vuole unificare nuovamente gli orchi e muovere guerra ai regni degli uomini che da sempre sono visti come oppressori degli orchi. Tradizione orale vuole che gli Orchi occupassero un tempo le pianure sotto alle montagne, dove la terra è fertile e produce frutti da sola.
Questo, se vero, è estremamente remoto nel tempo. Gli orchi hanno probabilmente subito l'espansione territoriale di uomini e nani, specie molto più sociali e sedentarie. Gruppi di (mezz')orchi nomadi permangono ancora nel continente, anche se molti tratti si sono stemperati col tempo per la loro interfertilità con gli umani.
Gli orchi di sangue più puro sono ormai piccoli enclavi isolati dove uomini e nani non sono troppo interessati a vivere. 

\subsection*{Eventuale cattura di Grok}
Se Grok verrà portato al campo principale l'interrogatorio avverrà alla presenza di Djen, che in tale situazione avrà i privilegi di un alto diplomatico e potrebbe magari essere il motivo per cui è garantita la presenza del party ad un interrogatorio sotto a zona di verità.
Grok renderà palese di ritenere quello contro gli orchi un genocidio, ma anche che farebbe lo stesso su uomini e nani avendone l'occasione. A domande sull'aver mai visto architetture nella pietra con i simboli sacri di huin questo sarà costretto a dire di sì, rivelando l'esistenza di quello che è noto agli orchi come Yar uk Dill, il pozzo delle stelle, dove i capiclan si radunano quando necessario.
Djen simpatizza generalmente per gli Orchi, e la loro tendenza a combattere frontalmente e con grande valore, ma tale immagine non è certo ritrovabile in Grok.

\subsection*{eventuale Fuga di Grok}
Se Grok dovesse essere in grado di allontanarsi potrebbe essere direttamente catturato da Djen, se il party raggiunge il campo come prima cosa (e viene magari convocato per aver combattuto contro il prigioniero). Altrimenti potrebbero magari incontrare il chierico i città.

\section{Djen, il chierico}
Djen proviene dal nord del continente, ed è un chierico di Huin, divinità dei nani a cui sono cari i combattenti e la prestanza fisica.
Si trova a sud della catena montuosa che prima o poi dovrà pur avere un nome per indagare sulla possibile presenza di un antico santuario nanico tra le montagne, per avviare un progetto internazionale di recupero che consenta al suo ordine di lavorare sul territorio del regno per restaurare tale santuario.
Questo santuario si troverà ben nel territorio degli orchi, e Grok potrebbe saperlo. Tale luogo potrebbe anzi essere usato per raduni delle tribù distinte degli orchi, e avere pertanto un nome orchesco (Yar uk Dill).
Grok chiaramente farà il possibile per difendere tale luogo, e un tentativo di conquistarlo sarà certamente sufficiente ad attirare tutte le tribù orchesche contro di sè, e quindi piuttosto difficile da portare avanti.

\section{Loran, il bastardo}
Magari se il party torna dritto in città con Charlie, potrebbero incontrare lungo il loro percorso i banditi di Loran. 
\end{document}