\section{Il Regno di Oster}

%completare i paragrafi su Hector e Dalhil

\subsection{Geografia e Società}
Il regno di Oster occupa la penisola di Rhuba ad alcune isole ad est di essa, coprendo all'incirca $250000Km^2$ e contando attorno ai $13$ Milioni di abitanti. I paesaggi che presenta sono piuttosto vari: dalle catene montuose orientali, alle pianure agricole del nord-est, comprendendo anche paludi, ampie foreste e paesaggi collinari.
La lunga costa della penisola ospita diverse città mercantili, figlie delle linee di commercio marittimo tra i continenti di Desthir, di cui il Regno fa parte, e Quaranur, a nord-ovest, separato dallo stretto di Talaniya.

La maggior parte del territorio è a vocazione agricola, e la maggior parte della popolazione è impegnata in questo settore. Soprattutto nel sud del paese, è raro che la popolazione possieda la terra che coltiva, e vive prevalentemente in umiltà, con le eccezioni di alcuni nuclei mercantili ed imprenditoriali come la città di Camir-Bahil, porta orientale del regno; Nebelport, la cui conurbazione occupa quasi tutta la punta occidentale del Bershilf e alcune aree manifatturiere diffuse nel meridione.
A livello territoriale la maggior parte di queste manifatture sono situate nei territori di Friedshaff, Caraidd meridionale e Deredon, e coincidono con le zone più densamente popolate del regno.

Nel Kirrimur ed il nord del Dwyndear vi sono le maggiori risorse minerarie del regno, soprattutto ferro, stagno e torba, ma anche rare vene di metalli preziosi.

Nelle porzioni più continentali di Bershilf ed Enedor viene invece prodotta la maggior parte del grano e del riso: in queste aree il suolo è un continuo alternarsi di ampi appezzamenti di terreno arido, inframezzati da macchie paludose causate dalle frequenti risorgive spontanee. Nel tempo, nella zona, è stato costruito un efficente e ramificato sistema di canali di irrigazione che consentono la coltivazione di diversi cereali in tutta la regione.

Durante la formazione del regno, relativamente giovane, la città di Andorbrig, già sede della Chiesa, venne scelta come capitale: in parte per serrare il controllo statale sulle attività del clero, ed in parte per la sua centralità nella regione, che oggi è la chiave della sua importanza negli scambi di merci all'interno del regno, oltre che del suo peso come cuore culturale del paese.

A livello culturale, a meno di rare eccezioni, Oster è un regno relativamente arretrato. La nobiltà locale ed il clero sono molto forti, e la classe borghese è poco presente, se non nel sud del paese.
La scolarizzazione della popolazione segue la stessa diffuzione geografica della borghesia, ed è spesso dipendente dal clero, per l'erogazione. Inoltre non è raro che i pochi borghesi delle aree centrali e settentrionali del paese manchino quasi completamente di educazione.

La presenza del Clero sul territorio è capillare e influenza notevolmente la vita pubblica, specie nelle aree pià periferiche. Esso è centrale dal punto di vista politico, proprio per questo forte radicamento culturale, tanto da avere, nel tempo, acquisito il controllo di interi feudi: Ad esempio l'Arconte di Caraidd è da generazioni un Vescovo della Chiesa.

\subsubsection{Titoli Nobiliari e Civili }
In cima alla gerarichia vi è il singolo sovrano, che porta il titolo di \textbf{Re} o \textbf{Regina} a seconda del sesso. Se il sovrano ha un consorte questo sarà immediatamente al di sotto di esso, in termini di etichetta, ma tale titolo non conferisce automaticamente ulteriore potere temporale, se non, tradizionalmente, la proprietà di una piccola cappella all'interno del perimetro del palazzo reale, che dai tempi della Regina Augusta viene offerta come dono nuziale dal sovrano al consorte.
Il consorte prende il titolo di \textbf{Principe Consorte} se maschio e di \textbf{Regina Consorte} se donna.

Al di sotto dei regnanti, in termini di etichetta, vi sono il \textbf{Principe o Principessa Ereditaria} (seppur il titolo sia tendenzialmente trasferito in linea maschile non è impossibile che sia assegnato ad una donna), ovvero un figlio o figlia biologica dei regnanti, primo in linea di successione, ed il \textbf{Principe o Principessa}. Questi titoli non danno automaticamente diritto ad una demesne o degli uomini, seppur sia comune che il Sovrano garantisca titoli secondari o uomini ai propri figli.

Appena al di sotto, in termini di etichetta, ma secondo solo al Sovrano in effettivo potere temporale vi è il titolo di \textbf{Arconte}, il cui dominio è detto Territorio. Scendendo ancora ci sono i titoli di \textbf{Conte} e \textbf{Marchese}, tecnicamente equivalenti, ma con un diverso tipo di accordo con il proprio signore: un Marchese deve versare meno tasse, ma è tenuto ad un maggior contributo militare alle forze del proprio signore, e ad una maggiore trasparenza amministrativa, rispetto al Conte. Questi titoli, seppur equivalenti in termini nobiliari, sono percepiti diversamente, ed un Marchese nutre tendenzialmente di maggior rispetto, in confronto alla sua controparte.

Scendendo ulteriormente di grado troviamo poi il titolo di \textbf{Visconte}, per i vassalli diretti di Conti e Marchesi, ed infine di \textbf{Barone}. In effetti esistono titoli ulteriori, di rango più basso, che non garantiscono il diritto ad una demesne.
Tra questi vi sono i titoli di \textbf{Cavaliere Ereditario}, generalmente conferito per il servizio o valore offerto in battaglia da alcuni individui, e quello di \textbf{Baronetto}, che in quanto ereditari, sono ancora considerati come titoli Nobiliari. Sostanzialmente di grado equivalente, secondo l'etichetta, sono i \textbf{Signori} che, tuttavia non hanno una carica ereditaria, nè, tecnicamente, permanente, seppur non sia raro che essa resti non solo ad un singolo individuo per tutta la vita, ma è anche comune che passi ad altri membri della stessa famiglia, alla morte del precedente detentore della carica.

A livello puramente Pratico i Signori sono più vicini ad un Barone che ad un Baronetto, e hanno tendenzialmente una carica civile di controllo su città di particolare peso locale. Eccezione al tendenzialmente ridotto potere di questo titolo è il signore di Nebelport, grande città commerciale nella punta continentale occidentale di Bershilf, il cui controllo su uno dei principali porti del regno ne aumenta notevolemente il peso politico.

In ultimo vi sono i titoli di \textbf{Cavaliere} e \textbf{Scudiero} che identificano Sostanzialmente tutti coloro che non rientrano strettamente nel volgo, in quanto rivestono, o hanno rivestito in vita, un ruolo importante dal punto di vista, tendenzialmente, militare.

\begin{DndSidebar}[float=!b]{Meta}
  Il primo giorno di campagna si svolge in data 17 agosto 968. In particolare si noti che Frida ha 17 (due anni più dell'età di matrimonio minima per legge) e suo fratello ha 12 anni.
\end{DndSidebar}

\subsection{Storia Recente in breve}
Il Regno di Oster nasce dalla lenta unificazione, in parte tramite legami famigliari, in parte tramite conquiste militari, della penisola di Rhuba. Già sede centrale della Chiesa e culturalmente piuttosto coesa, nonostante la presenza di alcune minoranze, specialmente nel sud, la penisola si unì sotto un unica bandiera solo nel 965, con l'annessione di Deredon per mano di re Gustav II.

Prima dell'annessione di quest'ultimo territorio il Regno aveva conosciuto circa 130 anni di relativa stabilità politica: il periodo più lungo senza conflitti nei precedenti cinque secoli, ricchi di conflitti di diversa magnitudine.

\subsubsection{Guerra di Successione Osteriana} %famiglie Wittel e Uller
%12 dicembre 834-24 giugno 835, incoronazione di Ulrich II wittel il 26 giugno
Di particolare peso nella storia del regno è la Guerra di Successione Osteriana. La principale premessa del conflitto fu la morte di Principe Oster Il Santo, grande protagonista delle guerre di unificazione dei primi dell'800 e fulcro di un triumvirato composto da Ulrich Wittel (poi passato alla storia come "il Vecchio" per distinguerlo dal figlio omonimo che diverrà Re, ma anche perchè già sessantenne durante le campagne di unificazione) e Lundovir Uller. Il triumvirato controllava le maggiori forze militari del paese e riunì tutti i territori della penisola, escluso il Deredon, sotto la bandiera di Oster.

In seguito alla morte di Oster, l'espansionismo militare si arrestò, anche per via delle problematiche amministrative che derivavano dalla gioventù dello stato e dalla precedente eterogeneità amministrativa dei Territori. Persa la spinta unificatrice della figura di Oster, iniziarono a crearsi attriti tra Ulrich il Vecchio e Lundovir, ma fu solo con la morte del primo, causata dall'età, che il triumvirato crollò definitivamente.

Ulrich il giovane, infatti, aveva un indole ben meno mansueta del padre, e vedeva l'immobilismo di quest'ultimo nei confronti di Lundovir, che egli considerava "un barbaro inadatto al comando", come una debolezza. Per questo motivo, una volta Arconte di Unador, egli tentò il colpo di stato: La mattina del 12 Dicembre 834 Ulrich sciolse ufficialmente il triumvirato e si dichiarà sovrano di Oster accentrando l'amministrazione nelle sue mani, ed affidando ad alcuni uomini fidati l'assassinio di Lundovir.

Nonostante la competenza della guardia personale della famiglia Wittel il tentativo di assassinio fallì, e la risposta di Lundovir e dei suoi alleati non si fece attendere. Oster, ancora giovane, si fratturò in due schieramenti contrapposti: Da una parte gli arconti di Unador, Friedshaff, Caraidd e Dwyndear, e dall'altra il potente Arconte di Kirrimur, ed i suoi alleati: gli arconti di Bershilf e Enador.

Fu una guerra breve: in primo luogo le forze di entrambi gli schieramenti erano sfiancate da anni di guerre, e i compagni d'arme di oggi erano stati nemici fino a meno di dieci anni prima. In secondo luogo una disastrosa sconfitta nei primi mesi del conflitto, subita dalle forze Kirrimuriane a Camir-Dwyn indebolì l'unica armata in grado di contrastare l'avanzata della coalizione di Ulrich.

Il 24 Giugno 835 Lundovir venne ucciso da una squadra di incursione durante l'assedio di Kirrcreag, la fortezza della famiglia Uller, ed insieme a lui i suoi tre figli, di 21, 17 e 4 anni. La mattina dopo il forte dichiarò la resa e la guerra terminò.

Fu il 12 Luglio dello stesso anno che Ulrich si incoronò come Ulrich II Wittel. La scelta di tale nome rispecchia il riconoscimento a posteriori del ruolo del padre, che nella narrativa nazionale sarà ricordato come primo re del paese.

Il titolo di Arconte di Kirrimur sarà affidato a Luthel Uller, cugino di Lundovir e appartenente ad un ramo cadetto della famiglia.


\subsubsection{Guerra Annessione Deredorniana}
%7 marzo 961 - 4 marzo 965
Il capitolo finale dell'unificazione della penisola sotto allo stemma Osteriano fu scritto dall'allora Re Gustav II. Dopo alcuni mesi di preparativi, il 7 marzo 961, Gustav attraversò il confine Deredorniano, senza dichiarare guerra, con la scusa diplomatica del non riconoscimento del governo di Deredon e il già teorico possesso di quelle terre da parte della corona.

Nonostante la mancanza di una dichiarazione di guerra, l'esercito Deredorniano non fu preso di sorpresa, e si oppose duramente all'avanzata delle forze Osteriane. In effetti voci dei preparativi del conflitto erano giunte anche alle orecchie degli uomini del sud, che si prepararono a ricevere l'invasione.

La guerra fu breve ma sanguinosa, e si svolse quasi completamente sul territorio Deredorniano. Ciò minò rapidamente le capacità produttive del paese, affamandolo fino al termine del conflitto.

Affatto secondario è il modo in cui la popolazione locale reagì all'invasione, facendo fronte comune e portando avanti anche azioni di guerriglia indipendenti dalle azioni dell'esercito regolare. Tutt'oggi dei nuclei indipendentisti minano la stabilità del Deredon.

Nonostante la disperata resistenza all'invasione il piccolo territorio del Deredon non potè nulla contro l'avanzata della potente armata Osteriana. Infatti questa era non solo più numerosa e meglio addestrata, ma disponeva anche di più armi e di qualità superiore, senza contare la disponibilità dei Chierici della Pace, che da soli rappresentavano un grosso ostacolo sul campo di battaglia.

Il 4 Marzo 965, Gustav II stesso entrò nella Capitale di Deredon, Arell, ormai occupata e pose fine alla guerra decapitando personalmente Micolash V, l'allora sovrano di quel territorio, accusandolo di essere un usurpatore di terre della corona, e dichiarando l'illegittimità di ogni pretesa della sua linea di sangue alla nobiltà. Il territorio di Deredon fu quindi affidato al fratello, Heinrich, che dovette però rinunciare al nome Wittel su pressione della maggiore nobiltà, che osteggiava un possibile accentramento di troppi territori nella demesne della corona.


\subsection{Famiglia Reale}
\subsubsection{Il Defunto Re Gustav II Wittel}
%(12 agosto 930 - 27 settembre 966) morto in un'imboscata di indipendentisti Deredorniani
Re Gustav II Wittel (12 agosto 930 - 27 settembre 966) governò per breve tempo: circa undici anni, molti dei quali spesi in guerra. Salì al governo in seguito all'abdicazione del padre avvenuta poco dopo la nascita di Hector. Quest'ultimo perì prima della campagna di unificazione Osteriana, e non poter mostrare al padre un regno unificato fu sempre un cruccio per Gustav.

Durante gli anni precedenti alla guerra fu supportato nell'azione di governo dal primo ministro che già collaborò con suo padre, ma che divenne sostanzialmente l'unico amministratore del paese durante il conflitto. Fu tutto sommato un governante mediocre, ma investì poca attenzione alla gestione del paese, aprendo così la strada ad una presa di potere da parte dei burocrati della Chiesa.

Guidare la campagna di riconquista del sud della penisola fu per lui sempre ben più imprtante di ogni altra cosa: non fu mai particolarmente vicino alla famiglia, nè così interessato alle grazie della moglie. Trascorse gran parte della sua vita di corte a contatto con gli alti gradi dell'esercito.
Morì a campagna militare ormai terminata, in seguito ad un'imboscata di un gruppo di indipendentisti Deredorniani, attraversando il passo di Tël con un ristretto gruppo di cavalieri.

\subsubsection{Regina Madre Elsa Maria Bruchen Von Friedshaff}
%(24 maggio 935 -)
Moglie di Gustav II e vedova di quest'ultimo. Donna devota e profondamente ambientata nel proprio strato sociale: crebbe i figli nella stretta osservanza del Credo senza mai lasciare nè far lasciare loro la corte (con l'eccezione del periodo di tutorato della figlia sotto al fratello), con particolare attenzione al figlio maschio, Hector. Nonostante, infatti, la legge preveda una trasmissione di eredità per primogenitura, è secolare costume che le figlie delle famiglie nobili vengano date in moglie con clausole di rinuncia ai propri titoli, cosa che all'atto pratico rende la trasmissione di titoli nobiliari e proprietà quasi esclusivamente patriarcale.

Durante il conflitto Oster-Deredorniano si dedicò all'educazione del figlio. Il fratello, Dalhil, non unitosi alla campagna a causa della sua disabilità, chiese ed ottenne il tutorato su Frida, primogenita della coppia reale, per il periodo di assenza del padre, in quanto consapevole del carattere e delle convinzioni della sorella minore. Nonostante situazioni simili non fossero rare, era inusuale che la richiesta riguardasse una figlia primogenita e non un figlio maschio cadetto. Ciononostante il sostanziale disinteresse nella figlia della Regina fece sì che tale richiesta venisse accolta. Sia per l'età di questa nel periodo di tutorato, sia per la freddezza della madre nei suoi confronti, Frida non ha particolare affetto per la madre, nè condivide le sue posizioni.

\subsubsection{Hector Wittel: Il giovane Monarca}
%(3 ottobre 955 - )
puppet sad boi

\subsubsection{Frida Wittel}
%(15 aprile 951 - )
Frida trascorse l'infanzia nella corte reale, venne istruita in modo appropriato al suo rango: apprese la lettura e la scrittura, l'etichetta ed il ricamo, e come far di conto. Durante l'infanzia non ebbe particolari contatti con i genitori: il padre era spesso assente ma gentile nei suoi confronti, mentre la madre, più presente ma ben poco coinvolta, specie dopo la nascita di Hector, era sempre stata severa e distaccata. Poco dopo la nascita di quest'ultimo alcuni contratti di matrimonio furono soppesati, su spinta della madre, ma Gustav non si dedicò mai alla cosa, e non arrangiò pubblicamente nulla di definitivo.

Fu solo all'età di dieci anni, durante il tutorato alla corte dello zio, che trovò affetto e spensieratezza, legandosi a questo più che al proprio padre e stringendo forti amicizie con la sua figlia minore, Teresa. Durante questa esperienza apprese come tirar di scherma, le arti della guerra e si dedicò alla letteratura contemporanea, scrivendo anche qualche verso, che rimase sempre privato. Imparò a cavalcare e venne a contatto con personaggi influenti, ma estranei alla vita di corte. La tendenza dello Zio ad osteggiare la chiesa ed il suo coinvolgimento nel potere temporale fu importante nella sua formazione, e plasmò le sue scelte.

Dopo la morte del padre ed il ritorno a corte continuò ad essere molto legata allo zio, corrispondendo spesso con lui e ricevendone consiglio, visitandolo ad ogni occasione possibile. Al suo ritorno a corte scoprì dell'esistenza di alcuni documenti, firmati dal padre prima dei suoi quindici anni, che la promettevano in sposa a [da definire, ma pezzo grosso in un territorio vescovile]; uomo più anziano e vicino alla chiesa. Tali documenti sarebbero stati infine firmati in concessione alla moglie durante il periodo del conflitto. Frida rifiutò categoricamente il contenuto di questi documenti, e sostenne si trattasse di falsi pur non potendo mai dimostrare la cosa. A causa di ciò fuggì e prese rifugio alla corte dello zio, attorno al quale si coagulò rapidamente una fazione anticlericale, figlia di tutti i dissapori che una nutrita schiera nobiliare nutriva nei confronti del crescente potere temporale del clero. Oggi il Monarca riconosciuto è il fratello Hector, sotto la guida della madre come regina reggente e dei suoi precettori legati al clero. Nonostante la teorica illegalità delle azioni sue e dello zio che la tutela, per non causare una guerra civile, si è pubblicamente minimizzata la faccenda, cercando di evitare il confronto diretto, che sembra tuttavia inevitabile. Prima di una rivendicazione del trono, od una pubblica condanna di Frida entrambe le fazioni vorranno essere certe del proprio vantaggio, ma dopo due anni di tensioni e numerose azioni occulte da ambo le parti, l'equilibrio di forze è instabile e difficile da valutare, e nessuno sa quando potrebbe scoccare la scintilla definitiva che infiammi le ostilità.

\subsubsection{Dalhil III Bruchen, Arconte di Friedshaff}
%(23 maggio 915 -)
Baddass ma si è rotto una gamba da giovane, cadendo da cavallo e ora zoppica.
In guerra va il suo primogenito al suo posto suo. La guerra è dal 61 al 65, il figlio è del 32 quindi ha 29 anni circa. facciamo un figlio di 29, uno di 25, e una figlia, Teresa, di 23.

%PRESENT TIME: 17 agosto 968

\clearpage
